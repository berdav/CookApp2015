Lo studio di un'interfaccia per un'applicazione mobile ci ha dimostrato come sia difficile
progettare applicazioni che si rivolgano alla popolazione in generale, con vari livelli di esperienza
e età molto differenti. Tuttavia, grazie alla metodologia di testing discount usability, si sono potute
effettuare rapide iterazioni dell'applicazione, nelle quali si sono risolti man mano i problemi evidenziati
dagli utenti. Tale approccio iterativo consente di verificare immediatamente se i successivi utenti evidenzieranno
gli stessi problemi emersi in precedenza, oppure se il problema risulta superato.
\\
Durante la fase di testing, in particolare, è emersa in modo chiaro l'importanza dell'utilizzo di simboli per
indicare all'utente quale azione venga effettuata all'interazione con un pulsante. Se in alcuni casi si è scelto di mantenere
il testo che descrive l'azione che un bottone effettua, in molti altri si è scelto di utilizzare solamente icone, in maniera
da migliorare la resa estetica dell'interfaccia, senza sacrificarne l'intuitività.
\\
Dopo varie iterazioni l'applicazione è risultata relativamente facile da usare per una base d'utenza variegata, dimostrando
la validità dell'approccio da noi adottato. Per ottenere più informazioni sarebbe però necessaria una vera e propria implementazione
dell'applicazione, che potrebbe venire usata per mostrare anche interazioni impossibili con un approccio a wireframe (es gesture, animazioni,
interazioni vocali).