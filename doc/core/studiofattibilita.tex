\subsection{Personas}
\hrulefill\\
\begin{tabular}{l | c}
	\textbf{Mirco Fabbri} & \\
	\textbf{Età:} & 19 anni\\
	\textbf{Professione:} & studente\\
	\textbf{Esperienza in cucina:} & Prime armi\\
	\textbf{Competenza tecnologica:} & Molto alta\\
\end{tabular}\\\\
	\textbf{Descrizione:}  Mirco è uno studente, per impegni di lavoro di
	entrambi i genitori tra poco rimarrà spesso a casa da solo, per questo
	sua madre ha deciso di insegnargli a cucinare assistendolo nei primi
	passi in questo mondo.  Usa sempre lo smartphone ma è molto impaziente
	nei confronti della tecnologia, se qualcosa non va come vuole lui si
	scalda subito e la getta via!\\

\hrulefill\\
\begin{tabular}{l | c}
	\textbf{Giulia De Felice} & \\
	\textbf{Età:} & 23 anni\\
	\textbf{Professione:} & studente\\
	\textbf{Esperienza in cucina:} & Medio-bassa\\
	\textbf{Competenza tecnologica:} & Medio-alta\\
\end{tabular}\\\\
	\textbf{Descrizione:}  Giulia è la classica studentessa abruzzese fuori
	sede, cucina bene ma non ha molti soldi, studia all'università ma tende
	a distrarsi spesso e usa tutti i giorni social e pc, anche se non si
	ritiene una grande esperta.  Usa il telefono perlopiù per giocare e
	messaggiare ma molte volte preferisce al contatto telematico una bella
	serata tra amiche.\\

\hrulefill\\
\begin{tabular}{l | c}
	\textbf{Cabdalle Sharif} & \\
	\textbf{Età:} & 26 anni\\
	\textbf{Professione:} & elettricista\\
	\textbf{Esperienza in cucina:} & Medio-alta\\
	\textbf{Competenza tecnologica:} & Medio-bassa\\
\end{tabular}\\\\
	\textbf{Descrizione:}
	Cabdalle è un'elettricista di origini africane (somalia), è felicemente
	fidanzato e vive con la sua ragazza. Purtroppo, per motivi di lavoro e
	non avendo orari fissi, non è quasi mai a casa e quindi cerca sempre di
	fare la spesa e cucinare nel minor tempo possibile.  È musulmano, per
	questo non beve alcoolici e non mangia carne di maiale.

\hrulefill\\
\begin{tabular}{l | c}
	\textbf{Roberto Sasdelli} & \\
	\textbf{Età:} & 34 anni\\
	\textbf{Professione:} & salumiere\\
	\textbf{Esperienza in cucina:} & Medio-bassa\\
	\textbf{Competenza tecnologica:} & Bassa\\
\end{tabular}\\\\
	\textbf{Descrizione:}
	Roberto è un salumiere, forse anche per questo motivo risulta essere una
	persona molto attenta alle materie prime e ai prodotti che mette nel suo
	frigorifero nonostante non sia un eccellente chef.  Una delle cose che
	più apprezza però è il buon vino, spesso si concede una pizza
	accompagnata da qualche rosso corposo. È single ma ha molti amici.

\hrulefill\\
\begin{tabular}{l | c}
	\textbf{Erica Paganelli} & \\
	\textbf{Età:} & 45 anni\\
	\textbf{Professione:} & impiegata\\
	\textbf{Esperienza in cucina:} & Alta\\
	\textbf{Competenza tecnologica:} & Bassa\\
\end{tabular}\\\\
	\textbf{Descrizione:}
	Erica è una mamma un po' sfortunata, divorziata da tempo dal marito
	cerca di cucinare sempre al massimo per i figli durante la settimana,
	poiché il weekend rimangono ospiti del suo ex marito, e per se stessa.
	Il lavoro non le consente troppo tempo per la cucina, ma quando può
	cerca sempre qualche ricetta alternativa per stupire e sperimentare. Fa
	la spesa dai vari commercianti, comprando specialmente prodotti a
	chilometro zero.

\hrulefill\\
\begin{tabular}{l | c}
	\textbf{Franz Lambertini} & \\
	\textbf{Età:} & 53 anni\\
	\textbf{Professione:} & medico\\
	\textbf{Esperienza in cucina:} & Molto bassa\\
	\textbf{Competenza tecnologica:} & Alta\\
\end{tabular}\\\\
	\textbf{Descrizione:}
	Franz

\hrulefill\\
\begin{tabular}{l | c}
	\textbf{Vanda Neri} & \\
	\textbf{Età:} & 75 anni\\
	\textbf{Professione:} & pensionata\\
	\textbf{Esperienza in cucina:} & Molto alta\\
	\textbf{Competenza tecnologica:} & Molto bassa\\
\end{tabular}\\\\
	\textbf{Descrizione:}\\

-Franz Lambertini, 53 anni, italo-tedesco. Primario di medicina. Vive con la moglie, ma suo figlio vive in un’altra città. Con la tecnologia è un disastro ed usa whatsapp solo per marcare stretto suo figlio. Sua moglie non lo aiuta troppo in cucina. 
-Vanda Neri, 75 anni, pensionata. Cucina quando può per i figli e i nipoti dai quali è stata costretta a usare lo smartphone, mondo che comunque la affascina. Ha fatto la casalinga per cui ha solo la licenza media ma è bravissima in cucina. Si diverte ad usare i social network perchè ha un figlio che vive all’estero. Ha un sacco di tempo per cucinare.

\subsection{Scenari}
1. Mirco è rimasto solo a casa per le vacanze natalizie. I suoi gli hanno lasciato il frigo vuoto. Mirco decide allora di invitare il suo amico Paolo a casa a vedere un film. Entrambi sono incapaci di cucinare, decidono quindi di usare l’applicazione per scegliere una semplice ricetta, fare la spesa e cucinarla. Scelgono di fare degli spaghetti alla carbonara, impostando le dosi per due. Durante la preparazione i due ragazzi combinano disastri uno dietro l’altro, per cui non possono stare attaccati al telefono, devono per ciò usare i comandi vocali, in modo da non distruggere la cucina e il telefono.
2. Giulia è ancora a Bologna perché ha avuto l’ultimo esame a Dicembre, l’ha passato con ottimi voti ma dovendo studiare non si è occupata troppo della spesa. Il suo frigo è quasi vuoto ma contiene uova e zucchine, decide quindi di fare una frittata. La sua amica Daniela la chiama autoinvitandosi, non volendo sentire ragioni sul poco cibo presente in casa. Giulia deve allora inventarsi un primo, perchè le uova non bastano per due persone. Decide allora di fare una pasta con le zucchine, per poi inventarsi qualcosa per finire le uova. Trova delle cipolle e decide di usare quelle per la frittata.
3. Cabdalle ha litigato con la sua ragazza e vuole scusarsi preparandole HILIB IYO MUFO (carne di vitella o manzo con il pane arabo “pita”) e un piatto che lei ama molto, una cheesecake ai frutti di bosco. Esce dal lavoro un po prima del solito e corre a fare la spesa. Decide di fare la spesa al supermercato ma non compra il vitello. Il suo macellaio di fiducia è chiuso, quindi usa l’app per trovare un macellaio islamico nelle vicinanze. 
4. Roberto dopo una lunga giornata di lavoro si rende conto che gli è rimasto un filetto di manzo. La carne è molto buona, per cui non vuole congelarla ma non è riuscito a venderla in tempo, per cui decide di portarsela a casa per mangiarla.  La carne è parecchia, per cui decide di invitare i suoi amici per mangiarla assieme. Propende per una preparazione alla griglia, ma non sa che vino e che dessert scegliere per la serata. Usa quindi l’app per farsi consigliare un vino da abbinare alla grigliata e un dolce.
5. È natale, quindi Erica decide di preparare qualcosa di elaborato per i suoi figli. Decide di fare delle tagliatelle al ragù di cinghiale, per cui deve trovare un posto dove lo vendano. Per compiacere i figli vorrebbe fare un hamburger. Si ricorda però durante la spesa che mangiare carne lavorata fa male, per cui decide di preparare un piatto vegetariano. I suoi figli però non amano troppo le verdure. Gli viene in mente guardando le ricette dell’app di fare un hamburger vegetariano.
6. La moglie di Franz è andata a trovare la suocera, che Franz non sopporta. Ha quindi deciso di rimanersene a casa. In un momento di malinconia si ricorda dei crauti preparati da sua mamma assieme alla salsiccia. Prova a chiamare sua mamma che però non gli risponde. Cerca allora con difficoltà un’applicazione per aiutarlo a cucinare e trova la ricetta dei crauti con salsiccia. Deve fare la spesa e cucinare.
7. Vanda ha visto i suoi nipoti la settimana prima, e li ha trovati molto deperiti. Ha deciso di invitarli a pranzo per strafogarli. La prima parte del menu è un classico: tortellini. Per curiosità apre l’applicazione per vedere che ricetta propongono per i tortellini in brodo. Legge inorridita le dosi e la lista degli ingredienti, che sono secondo lei sbagliate. Decide allora di lasciare un commento sulla ricetta evidenziando gli errori. Si vuole cimentare in un dolce che gli faceva sua nonna, ovvero il salame di cioccolato, che però non ha mai preparato. Cerca allora la ricetta e la reinterpreta a suo modo, lasciando poi un commento sulle sue modifiche.
