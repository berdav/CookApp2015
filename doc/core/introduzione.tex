Nel moderno contesto tecnologico, nel quale gli smartphone risultano essere molto
diffusi nella società, è interessante lo studio di strumenti software di ausilio
alle faccende domestiche. Grazie alla portabilità di tali device vi sono evidenti
vantaggi rispetto all'utilizzo di un laptop tradizionale, che, a differenza degli smartphone, non può essere
manovrato comodamente mentre si è impegnati in qualche operazione. Pertanto si è scelto di sviluppare l'interfaccia
di un applicazione mobile per smartphone che aiuti l'utente in cucina.
\\
A causa dell'emergere dei social network come strumento principale di interazione online fra utenti, si è
deciso di implementare, a partire da un'identità forntita da un social network esistente, funzionalità sociali
all'interno dell'applicazione. Tali funzionalità sono utili a gestire invitati ad una cena, richieste di aiuto
agli altri utenti dell'applicazione, etc.
\\
Oltre alle funzionalità sociali suddette, un altro aspetto innovativo dell'applicazione da noi studiata riguarda
la possibilità di introdurre suggerimenti predittivi all'interno dell'interfaccia. Se, ad esempio, si è pianificata una cena,
l'applicazione nelle schermate suggerirà gli ingredienti da acquistare o evidenzierà il prossimo passo da eseguire durante
la cucina vera e propria. Tale tipologia di interazione ``predittiva'', pur correndo il rischio di risultare invasiva,
è molto utile per aiutare gli utenti nei loro task quotidiani.