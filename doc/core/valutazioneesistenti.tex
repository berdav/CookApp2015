Costruire un sistema avente come punto cardine l'usabilità risulta un progetto
estremamente ambizioso senza avere punti di riferimento, per questo è stato
valutato l'utilizzo di un discount usability test.  Sono state selezionate
diverse applicazioni e diversi soggetti uniformemente distribuiti lungo gli
scaglioni identificati al termine della fase precente.  Andiamo ad indicare
quindi i dettagli della fase di valutazione delle applicazioni esistenti.

\subsection{Identificazione del test}
\subsection{Soggetti del test}

\subsection{Applicazioni utilizzate per il test comparativo}
Sono state selezionate tre applicazioni mobili disponibili sui principali market
e testate utilizzando il sistema operativo Android.  Si riporta la lista delle
suddette applicazioni:
\begin{itemize}
	\item Piccole ricette, piccola applicazione avente una base d'utenza
	limitata e un numero rilevante di ricette (circa 1800).
	\item GialloZafferano, applicazione mobile dai gestori dell'omonimo
	sito, popolare e con base di ricette notevole (circa 3400).
	\item Marmiton, applicazione francese con una grande base d'utenza e
	un massiccio numero di ricette disponibili (circa 64.000).
\end{itemize}

\subsection{Piccole Ricette}
\subsection{GialloZafferano}
\subsection{Marmiton}

\subsection{Task e risultati dei test}
