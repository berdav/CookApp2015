% test
% TODO: descrivi comandi vocali
Considerando le critiche elencate in precedenza e il target d'utenza rilevato si
presenta la seguente proposta di progetto, la quale costituirà la base per
un'eventuale applicazione conforme alle principali norme e linee guida a
riguardo dell'usabilità.

\subsection{Architettura dell'informazione}
Per la realizzazione del progetto si è utilizzato un approccio goal-oriented, top down.
Tramite questo approccio il sistema viene rifinito in modo progressivo a partire dagli
obiettivi inizialmente individuati.\\
Il progetto utilizza le linee guida e le interfacce base conformi al moderno
standard \emph{Material design}. Tale linguaggio di design si basa su una semplice
metafora: lo schermo bidimensionale viene pensato come tridimensionale, con oggetti
che hanno una posizione spaziale anche sul terzo asse, in modo da emulare una superficie
che possa creare e muovere volumi. Vi è poi un uso costante di animazioni a supporto di tale metafora e che
aiutano l'utente a individuare l'effetto delle proprie azioni.
Un altro aspetto importante del linguaggio è la semplicità
delle icone e dei widget: se negli anni passati si è assistito ad un enorme successo di interfacce
scheumorfiche\footnote{un interfaccia si dice skeumorfica se emula l'estetica di un oggetto reale}
, soprattutto grazie all'influenza di IOS nelle interfacce mobili, oggi si assiste
ad un ritorno a schermate più semplici, che non utilizzano metafore di oggetti reali.
Per quanto riguarda le funzionalità, si è scelto di implementare
molte funzioni ausiliarie, ma completamente integrate nel processo
culinario, il quale fa da padrone per l'intero utilizzo dell'applicazione.  Esse
rimangono completamente opzionali e più o meno invasive, a seconda della gravità
dei problemi che possono scaturire da un eventuale noncuranza delle
segnalazioni.\\
La ricerca di ricette, che siano esse nuove o già parzialmente conosciute, è una
parte molto importante nello sviluppo di un applicazione di supporto alla
cucina, per questo motivo è stata implementata una soluzione molto potente, rendendo
possibile per l'utente combinazioni complesse di filtraggio dei dati.\\
Si è voluto mantenere un concetto di accessibilità nell'applicazione, donando
all'utente opzioni per superare eventuali ostacoli dovuti a piccoli deficit.\\
Considerando il requisito riguardante il silenziamento di tutte le notifiche
dell'app stessa si è progettato il sistema per funzionare in modalità silenziosa
nel caso in cui il device sia rivolto verso il basso, rendendo il silenziamento
delle notifiche molto più semplice e immediato.  Per non risultare invasiva,
tale funzionalità è disattivabile dalle opzioni dell'applicazione.

\subsection{Design dell'interazione}

\subsubsection{Blueprint}
Queste aree sono state divise in varie sotto-schermate corrispondenti alle
singole iterazioni possibili dall'utente.  In figura \ref{fig:blueprint}
è riportato il blueprint relativo alle interazioni possibili tra le varie schermate appena descritte.
\begin{figure}[H]
	\centering
	\caption{Blueprint delle interazioni tra finestre}
	\includegraphics[width=\textwidth]{img/ipc_blueprint}
	\label{fig:blueprint}
\end{figure}
Esistono anche interazioni non specificate nel bluprint poiché troppo specifiche
alla sequenza delle interazioni, esse sono:
\begin{itemize}
	\item Premendo il tasto ``indietro" la schermata passa, ovviamente, alla
	schermata precedente, e così a ritroso fino alla schermata Home.
	Premendo nuovamente il tasto indietro (ove disponibile) in questa
	schermata verrà chiusa l'applicazione.
	\item La notifica contestuale interna all'applicazione non viene
	indicata all'interno del blueprint.
	\item La ricerca è considerata come contestuale, nell'area social
	verranno cercati gli utenti, mentre nelle altre aree verranno cercati i
	singoli piatti (rimandando alla schermata di ricerca avanzata)
	o gli ingredienti.
	\item È possibile modificare la dimensione del testo con gesti \emph{pinch to
	zoom}, la barra presente nelle impostazioni reagirà di conseguenza.
\end{itemize}

\subsubsection{Comandi vocali}
\begin{figure}[H]
	\includegraphics[width=\textwidth]{img/ipc_interazione_audio}
	\caption{schema relativo alle interazioni vocali in linguaggio naturale}
	\label{fig:ipcvocali}
\end{figure}
Sono stati implementati diversi comandi vocali.  L'interazione è iniziata
con il comando vocale "hey cookapp!" ed è in grado di comprendere diverse frasi
espresse in linguaggio naturale, si riporta in seguito la lista dei comandi
possibili, associati ad uno schema riassuntivo in figura \ref{fig:ipcvocali}\\
Dalla suddetta schermata � possibile invocare un comando per raggiungere una
determinata schermata, agendo di fatto come il men� per le interazioni
canoniche.  Ogni azione viene poi intrapresa contestualmente al compito e alla
schermata di partenza, in questo modo � possibile eseguire anche task complessi
utilizzando solamente l'interazione vocale.

\subsubsection{Wireframe}

\begin{quote}
 \textbf{Homepage}
\end{quote}
Il punto di partenza dell'interazione con l'applicazione è la home page. Tale pagina
si pone l'obiettivo di aiutare l'utente nella scoperta di nuove ricette e, durante
i primi utilizzi, delle funzionalità fondamentali. Si compone di una griglia di ricette
consigliate all'utente in base ai suoi pattern di utilizzo dell'applicazione.
Dalla home page è possibile accedere alla ricerca ed al menù principale, in modo da consentire
un rapido accesso alle funzionalità principali già dall'avvio.

\begin{quote}
 \textbf{Menù di navigazione}
\end{quote}
Il menù di navigazione, in modo analogo a quanto avviene nelle applicazioni che utilizzano material design, 
è costituto da un pannello sovrapposto rispetto alle schermate sottostanti e che può venire invocato tramite
un tasto menù o uno swipe dall'estremo sinistro dello schermo. Il menù consente l'accesso ai preferiti, all'elenco dei
menù dell'utente, all'area social, alle impostazioni ed alla lista della spesa, nonché alla pagina iniziale. È infine
possibile accedere al proprio profilo social tramite il proprio avatar.

\begin{quote}
 \textbf{Ricerca ricette}
\end{quote}
La ricerca delle ricette fornisce un potente pannello che consente di applicare vari filtri.
È infatti possibile escludere ingredienti dalle ricette, selezionare solo alcune portate fra antipasti, primi, 
secondi, dolci e piatti unici. È infine possibile cercare ricette di una certa categoria (vegetariane, senza glutine, etc).

\begin{quote}
 \textbf{Area Social}
\end{quote}
L'area social contiene le funzionalità relative all'interazione fra gli utenti. L'account social viene
mutuato da un famoso social network, dal quale è possibile modificare il proprio profilo.
È possibile condividere ricette, richiedere aiuto per ingredienti mancanti e passi
della ricetta. Gli altri utenti possono poi commentare i post che vengono pubblicati.

\begin{quote}
 \textbf{Gestione dei Menù}
\end{quote}
L'applicazione utilizza il concetto di menù per consentire di gestire ed organizzare pasti, gestendo gli 
invitati e la pianificazione temporale. Da tale sezione è possibile iniziare a cucinare: in questa modalità vengono
mostrati i passi completati e quelli mancanti per ciascuna ricetta selezionata. Aggiungendo utenti che sono presenti
sul social, se essi sono allergici viene mostrato un avviso per ciascuna ricetta che contiene l'allergene, in modo da consentirne
la modifica o la sostituzione.

\begin{quote}
 \textbf{Lista della spesa}
\end{quote}
Tramite la lista della spesa è possibile spuntare gli elementi che si sono già acquistati, richiedere direzioni verso negozi e supermercati,
utilizzare NFC o fotografare il codice a barre degli ingredienti per selezionarli e richiedere un ingrediente agli amici tramite il social.

\begin{quote}
 \textbf{Modalità cucina}
\end{quote}
Nella modalità cucina i menù mostrano il progresso della ricetta, ed un tocco sulle singole ricette conduce al passo corrente della ricetta. I singoli passi
contengono immagini e video contestuali. È scelta del redattore della ricetta se utilizzare in primo piano un'immagine o un video. È disponibile un timer
nel caso di operazioni che richiedano ad esempio una cottura o un periodo di attesa. Ciascun passo può essere condiviso sul social per richiedere consigli o aiuto.

\begin{quote}
 \textbf{Impostazioni}
\end{quote}
La schermata delle impostazioni permette di modificare la dimensione del testo, gestire le notifiche e la modalità ''non disturbare``. Tale modalità permette
di disabilitare l'audio e le notifiche, in modo da evitare fastidi all'utente. È inoltre possibile attivare questa modalità rivolgendo il telefono verso il basso.

\begin{quote}
 \textbf{Notifica contestuale}
\end{quote}
Durante il normale utilizzo del programma vi è la possibilità che appaiano sullo schermo degli aiuti contestuali da parte dell'applicazione.
Lo scopo di tali notifiche è quello di ridurre il carico cognitivo dell'utente, aiutandolo a ricordarsi di alcuni eventi in modo appropriato.
Tali notifiche sono di vari tipi: 
\begin{itemize}
 \item nel caso di menù già pianificati viene ricordato all'utente di iniziare a cucinare nel caso in cui il pasto sia prossimo;
 \item se si deve effettuare la spesa viene mostrato un ingrediente ancora non acquistato;
 \item vengono notificate le richieste di aiuto degli amici;
 \item nella schermata relativa al menù vengono mostrati piatti e bevande consigliati;
 \item se si sta cucinando, viene mostrato il prossimo passo da eseguire.
\end{itemize}
Le notifiche appaiono nella porzione in basso dello schermo durante l'esecuzione, mentre sono trasmesse al sistema operativo nel caso l'applicazione
sia inattiva. Tramite tap sulle notifiche si viene condotti alla pagina appropriata. È infine possibile effettuare uno swipe per nasconderle.
