In questa sezione verranno presentati il protocollo ed i risultati della valutazione
dell'applicazione CookApp. Tale valutazione è finalizzata ad ottenere informazioni su
eventuali errori presenti nell'applicazione, ordinandoli per gravità e frequenza. Vengono
inoltre ottenute informazioni generali sull'impressione degli utenti nell'uso dell'applicazione,
rispetto alla sua usabilità.

\subsection{Il protocollo di test}
La modalità di test scelta è discount usability testing. Tale modalità risulta essere particolarmente
adatta, in quanto è sufficientemente snella da poter essere applicabile nel nostro contesto.
Verrà inoltre utilizzata la modalità ``thinking aloud'', nella quale gli utenti sono invitati a esplicitare
verbalmente le sensazioni e le problematiche che emergono durante l'utilizzo dell'applicazione.\\
Sono stati selezionati per i test 10 partecipanti fra conoscenti, amici e parenti in modo da coprire al meglio lo spettro
demografico individuato durante la fase di analisi etnografica, in particolare rispetto all'abilità culinaria e tecnologica.\\
Dopo il completamento di ciascun task si richiede ai partecipanti di valutare l'intuitività dell'operazione effettuata in una scala
Likert di valori compresi fra 1 e 5. Come descritto in precedenza, sono previsti valori nulli nel caso in cui non sia stato
possibile completare l'operazione. Oltre all'autovalutazione degli utenti, sarà compito dell'intervistatore annotare i suoi pensieri
ed eventuali errori e golfi fra l'intervistato e l'applicazione.
I task sono stati selezionati per esplorare in maniera esaustiva tutte le funzionalità principali offerti dall'applicazione.
Vengono di seguito elencati i task scelti:
\begin{enumerate}
 \item Cerca una ricetta per nome;
 \item Cerca una ricetta per categoria;
 \item Condividi una ricetta;
 \item Leggi i commenti di una ricetta;
 \item Lascia un commento su una ricetta;
 \item Crea un menù;
 \item Trova una ricetta o un vino abbinati ad un menù;
 \item Trova istruzioni video relative ad un passo della ricetta;
 \item Aggiungi ingredienti di una ricetta alla lista della spesa;
 \item Trova un supermercato per fare la spesa;
 \item Spunta un elemento acquistato dalla lista;
 \item Chiedi aiuto per un ingrediente della spesa;
 \item Modifica la dimensione del testo;
 \item Attiva la modalità ``non disturbare'';
 \item Aggiungi un utente allergico al menù, modifica la ricetta che crea problemi;
 \item Inizia a preparare un menù e chiedi aiuto per un passo della ricetta.
\end{enumerate}

